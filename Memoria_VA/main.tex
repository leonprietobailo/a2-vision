%%%%%%%%%%%%%%%%%%%%%%%%%%%%%%%%%%%%%%%%%%%%%%
% To select a journal, use its code for the 
% journal= option in the \documentclass command.
% The journal codes for this template are:
% 
% Annals of Actuarial Science: aas
% British Journal of Political Science: jps
% Network Science: nws
% Political Analysis: pla
% Political Science Research and Methods: ram
% Evolutionary Human Sciences: ehs
% Natural Language Processing: nlp
%%%%%%%%%%%%%%%%%%%%%%%%%%%%%%%%%%%%%%%%%%%%%%
\documentclass[
  journal=medium,
  manuscript=article-type,
  year=2026,
  volume=33
]{cup-journal}

\usepackage{amsmath}
\usepackage[nopatch]{microtype}
\usepackage{booktabs}
\usepackage{parskip} 
\usepackage{enumitem}
\setlist[itemize]{left=1em}
\usepackage{titlesec}
% Espacio después de los títulos
\titlespacing*{\section}{0pt}{0.6\baselineskip}{0.6\baselineskip}
\titlespacing*{\subsection}{0pt}{0.5\baselineskip}{0.5\baselineskip}
\titlespacing*{\subsubsection}{0pt}{0.4\baselineskip}{0.4\baselineskip}

\title{Visión Artificial - Actividad 2: Exploración de filtros espaciales y morfológicos en escenarios reales}

\author{Orejón Blázquez, Álvaro}
\affiliation{Estudiantes, Universidad Internacional de La Rioja}
\email[Orejón Blázquez, Álvaro]{email@email.dom}

\author{Lopez Chamarro, Pau}
\affiliation{Estudiantes, Universidad Internacional de La Rioja}
\email[Lopez Chamarro, Pau]{email@email.dom}

\author{Iglesias Traviesa, Nuria}
\affiliation{Estudiantes, Universidad Internacional de La Rioja}
\email[Iglesias Traviesa, Nuria]{email@email.dom}

\author{Prieto Bailo, León Enrique}
\affiliation{Estudiantes, Universidad Internacional de La Rioja}
\email[Prieto Bailo, León Enrique]{contact@leonprieto.com}
% \author{T. Author}
% \affiliation{Second Division, Organization, City, Pincode, State, Country}

% \author{F.T. Author}
% \affiliation{Fourth Division, Organization, City, Pincode, State, Country}

\addbibresource{bibliography.bib}

\keywords{keyword entry 1, keyword entry 2, keyword entry 3} %% First letter not capped

\begin{document}

\begin{abstract}
Insert abstract text here. Lorem ipsum dolor sit amet, consectetur adipiscing elit, sed do eiusmod tempor incididunt ut labore et dolore magna aliqua. Lorem ipsum dolor sit amet, consectetur adipiscing elit, sed do eiusmod tempor incididunt ut labore et dolore magna aliqua. Lorem ipsum dolor sit amet, consectetur adipiscing elit, sed do eiusmod tempor incididunt ut labore et dolore magna aliqua. 
\end{abstract}

\section{Introducción}
\label{sec:intro}

El procesamiento digital de imágenes es una disciplina fundamental que permite mejorar, analizar e interpretar entornos reales. En campos donde se utilizan imágenes digitales, como la industria y la medicina, la aplicación de filtros
espaciales y morfológicos se ha convertido en una herramienta esencial para optimizar la calidad de las imágenes y extraer información relevante. Esto se debe a que las imágenes capturadas suelen estar afectadas por ruido, baja resolución y condiciones de iluminación,
lo que dificulta su análisis. La aplicación de estos filtros mejora la calidad visual de las imágenes, facilitando así su análisis y la extracción de información para la toma de decisiones.

Los filtros espaciales actúan modificando la intensidad de los píxeles en función de su vecindad, permitiendo así reducir ruido, suavizar regiones homogéneas y realzar o detectar bordes. Por otra banda, los filtros morfológicos actúan sobre la forma y estructura de los objetos mediante 
operaciones como la erosión, dilatación, apertura o cierre. Estas técnicas permiten eliminar elementos no deseados, separar o unir regiones y mejorar contornos.

Este trabajo presenta un enfoque multitemático que aplica, mediante un pipeline, un conjunto de filtros espaciales y morfológicos sobre imágenes de contextos médico, satelital e industrial. Este proceso permite comparar de forma directa los efectos de cada técnica dependiendo 
de la imagen utilizada. Además, se analiza cómo estas herramientas influyen en la calidad visual, la preservación de estructuras y la interpretabilidad de la información, identificando así las ventajas y limitaciones de cada filtro. De esta manera, se puede afirmar que 
no existe un filtro universalmente óptimo, y que la eficacia de cada técnica depende del contexto y de las características específicas de cada imagen.

\section{Material y métodos}
\label{sec:mat}

\subsection{Software utilizado}
\label{subsec:software}

El desarrollo se llevó a cabo en \textbf{Jupyter Notebook} (\textbf{Python}) ejecutado desde \textbf{Visual Studio Code}, mediante librerías ampliamente utilizadas para garantizar una implementación altamente reproducible.
Este entorno permitió integrar en un mismo flujo de trabajo la carga de imágenes, la aplicación de los algoritmos de filtrado y la visualización de los resultados.
La posibilidad de visualizar de manera consistente las imágenes originales y las procesadas facilitó la comparación de las distintas transformaciones aplicadas.

En particular, se empleó \textbf{OpenCV} (\texttt{cv2}) para la lectura de imágenes, conversión de color y la implementación de los filtros espaciales y morfológicos. 
Las operaciones de apoyo (preprocesado y postprocesado) se realizaron con la librería \textbf{NumPy}.
Por último, \textbf{Matplotlib} se utilizó para la generación de figuras y la comparación sistemática de resultados.

Para asegurar aún más la \textbf{reproducibilidad}, el procedimiento se diseñó de forma que los resultados sean replicables bajo las mismas condiciones de ejecución. 

\subsection{Selección de imágenes}
\label{sec:imagenes}

Con el objetivo de evaluar el comportamiento de filtros espaciales y operaciones morfológicas en un escenario multitemático, 
se trabajó con un conjunto acotado de imágenes reales (10 imágenes) seleccionadas de tres ámbitos diferenciados: \textit{industrial}, \textit{médico} y \textit{satelital/ambiental}. 

La selección se realizó siguiendo criterios orientados a maximizar la diversidad, es decir, priorizando ejemplos representativos con propiedades visuales claramente distintas. 
Al aplicar distintos filtros a cada una de las imágenes, se hizo posible la comparación entre las técnicas en condiciones cambiantes.

Estas imágenes fueron extraídas de tres fuentes principales, cada una para un tipo de escenario:
\begin{itemize}
    \item \textbf{Open-i (escenario médico).} Open-i es un dataset de imágenes biomédicas mantenido por la \textit{National Library of Medicine}, que nos permitió obtener imágenes relacionadas con la temática biomédica. 
    En esta práctica se empleó para seleccionar radiografías, donde predominan gradientes suaves y estructuras anatómicas con bordes definidos, lo que resulta extremadamente adecuado para analizar la atenuación de ruido mediante suavizado y los detectores de borde.

    \item \textbf{Science Source (escenario industrial).} Science Source es una biblioteca científica con imágenes relacionadas con tecnología, laboratorio e industria. 
    Se utilizó como fuente para seleccionar escenas industriales con patrones geométricos y contornos de alto contraste, idóneas para aplicar el filtrado espacial que afecta al detalle fino y las operaciones morfológicas, que pueden regular este tipo de estructuras.

    \item \textbf{EOSDA LandViewer / Sentinel Playground / Kaggle (escenario satelital y ambiental).} Se recurrió a plataformas de exploración ambiental y de imágenes satelitales.
    Estas nos proporcionan imágenes con grandes regiones homogéneas y elementos lineales (carreteras), lo que facilita comparar el comportamiento de los filtros cuando se enfrentan a variaciones de escala y contenido.
\end{itemize}

\subsection{Filtros espaciales}
\label{sec:esp}

Los filtros espaciales son operadores fundamentales en el procesado digital de imágenes y la visión por computador los cuales modifican las intensidades de los pixeles en base a los vecinos locales. A diferencia de otros métodos basados en dominios frecuenciales, el filtro espaciado se realiza en el dominio de la imagen, donde la salida de cada pixel es calculada como una función a partir de los pixeles vecinos definida por una mascara espacial o kernel. 

Formalmente, un filtro espacial puede ser descrito como una función que desliza una pequeña matriz (kernel) sobre la imagen, calculando un nuevo valor de píxel en cada ubicación a través de una operación predefinida. En el caso de los filtros espaciales lineales, esta operación corresponde a una convolución o correlación entre la imagen y el kernel. Los filtros espaciales no lineales, por otro lado, se basan en operaciones no lineales como la selección de mínimos, máximos o medianas dentro de la vecindad de pixeles.

Los filtros espaciales se categorizan habitualmente en función de los objetivos que persiguen. Los filtros de suavizado (o paso bajo) buscan reducir el ruido y pequeñas variaciones de intensidad promediando el valor de los píxeles en la vecindad. Ejemplos típicos incluyen filtros Gaussianos y de media, ampliamente utilizados como paso de preprocesado para mejorar la calidad y robustez de la imagen en análisis posteriores. Por el contrario, filtros como el realce de bordes enfatizan cambios de intensidad locales, resaltando bordes y detalles finos. Estos filtros son particularmente útiles para la extracción de características, detección de objetos y tareas de segmentación.

La efectividad de un filtro espacial depende de varios factores, incluyendo el tamaño y forma del kernel, el tipo de operación aplicada, y las características de la propia imagen. Un kernel excesivamente largo puede suavizar excesivamente estructuras importantes, mientras que un filtro de realce inadecuado puede amplificar ruido e introducir artefactos. Por lo tanto, la selección de filtros espaciales debe ser adaptada cuidadosamente en función de la aplicación y el contexto de la imagen. 

En escenarios prácticos, el filtrado espacial juega un rol crucial en diversos escenarios como la fotografía medica, la inspección industrial, la teledetección, y el análisis de imágenes biológicas. Al mejorar estructuras relevantes y mitigando variaciones indeseadas, los filtros espaciales facilitan la interpretación de información visual y mejoran el rendimiento de los algoritmos de visión por computador.

\subsubsection{Suavizado}
\label{sec:suavizado}

Los filtros de suavizado, también conocidos como filtros paso bajo, son una clase de filtros espaciales cuyo objetivo principal consiste en atenuar variaciones rápidas de intensidad de las imágenes las cuales están normalmente asociadas con ruido y detalles de alta frecuencia. Como resultado, los filtros de suavizado se emplean habitualmente como una de las etapas del preprocesado para la mejora de la calidad y para mejorar la robustez de tareas posteriores como la segmentación, detección de bordes y extracción de características. 

Conceptualmente, los filtros de suavizado funcionan agregando información de intensidad dentro de una vecindad local alrededor de cada pixel. Combinando los valores de la vecindad, se logra la atenuación de cambios de intensidad abruptos y se mantienen los cambios suaves en las regiones. Este proceso, conlleva una reducción de ruido pero introduce inevitablemente un cierto grado de desenfoque, particularmente en las fronteras de los objetos presentes. Consecuentemente, la aplicación del filtro paso-bajo implica un \textit{trade-off} entre la supresión de ruido y la conservación de los bordes. 

Los filtros de suavizado pueden ser clasificados entre lineales y no lineales. Los filtros lineales calculan la salida de cada pixel como la suma ponderada de las intensidades de la vecindad. El filtro mas sencillo es el filtro de media, el cual reemplaza cada pixel con la media aritmética de los pixeles circundantes. Si bien es eficaz para reducir ruido Gaussiano, el filtro de media tiende a difuminar los bordes y las estructuras finas. El filtro Gaussiano representa representa una aproximación mas refinada, en la cual los pesos siguen una distribución Gaussiana centrada en el pixel de interés. Este esquema de ponderación enfatiza los pixeles mas cercanos y resulta en un filtrado mas suave y natural, a la vez que ofrece un mejor control sobre el grado de suavizado. 

Los filtros de suavizado no lineales no se basan en operaciones de promediado, sino que aplican transformaciones basadas en estadísticas de orden o reglas. El filtro de mediana es una de las técnicas de suavizado no lineal mas utilizadas. Sustituye cada pixel con el valor de la mediana de las intensidades de la vecindad, haciéndolo particularmente efectivo en la supresión de ruido impulsivo, como el sal y pimienta. Otra ventaja importante de los filtros de media es su habilidad de preservar ejes y fronteras mas efectivamente que los filtros de media. 

El rendimiento de los filtros de suavizado está muy influenciado por el tamaño y forma del kernel, así como las características de ruido presentes en la imagen. Kernels mas pequeños proveen de reducción de ruido limitada a la vez que preservan bien el detalle, mientras que kernels mas grandes generan efectos de suavizado mas fuertes a cambio de la perdida de información estructural. Por lo tanto, la selección del filtro y del tamaño del kernel apropiado debe ser adaptado a la aplicación especifica y el contexto que presenta la imagen. 

\subsubsection{Realce de bordes}
\label{sec:rb}

\subsection{Filtros morfológicos}
\label{sec:morf}

\subsubsection{Dilatacion}
\label{sec:Dilatacion}
\subsubsection{Erosión}
\label{sec:erosion}
\subsubsection{Apertura}
\label{sec:ap}

La apertura es una operación morfológica que combina de forma secuencial una erosión seguida de una dilatación, utilizando el mismo elemento estructurante en ambas etapas. En la primera fase, la erosión reduce las regiones brillantes de la imagen y elimina 
pequeñas estructuras o detalles no deseados que no cumplen con el tamaño del elemento estructurante. Esto permite eliminar ruido o irregularidades sin afectar a los objetos principales de la imagen.

A continuación, la dilatación restaura parcialmente el tamaño de los objetos que han sobrevivido a la erosión, manteniendo su forma y sus contornos originales. De esta manera, la operación de apertura suaviza los bordes de los objetos principales 
mientras elimina los elementos de menor tamaño, modificando la estructura de la imagen de forma controlada.

El proceso de apertura actúa principalmente sobre elementos cuya dimensión es menor que la del elemento estructurante, afectando sobre todo a pequeños detalles o uniones entre objetos. El objetivo de esta técnica es obtener imágenes con estructuras más 
limpias y contornos más definidos, facilitando la interpretación y análisis de los objetos presentes en la imagen.

\subsubsection{Cierre}
\label{sec:cierre}

El cierre es una operación morfológica que combina de manera secuencial una dilatación seguida de una erosión, utilizando el mismo elemento estructurante en ambas etapas. En la primera fase, la dilatación expande las regiones brillantes de la imagen, 
rellenando pequeñas separaciones y conectando objetos cercanos que podrían estar ligeramente separados.

A continuación, la erosión reduce parcialmente el tamaño de los objetos que fueron ampliados durante la dilatación, restaurando sus formas originales y manteniendo las conexiones establecidas entre ellos. De esta manera, el cierre permite suavizar los bordes de 
los objetos principales, rellenar huecos y unificar regiones.

El proceso de cierre actúa principalmente sobre elementos más pequeños que el elemento estructurante, afectando a huecos en la imagen mientras conserva la geometría general de los objetos de mayor tamaño. El objetivo de esta técnica es obtener imágenes más uniformes 
y continuas, facilitando la interpretación y el análisis de las estructuras presentes.

\section{Resultados y métricas}
\label{sec:result}

\subsection{Escenario Industrial}
\subsection{Escenario Médico}
\subsection{Escenario Satelital}

\section{Conclusiones}
\label{sec:conc}

\section{Declaración de la IA}


























% ********* DEMO *********
\pagebreak
\pagebreak
\section{Insert A head here}
This demo file is intended to serve as a ``starter file''. It is for preparing manuscript submission only, not for preparing camera-ready versions of manuscripts. Manuscripts will be typeset for publication by the journal, after they have been accepted.

By default, this template uses \texttt{biblatex} and adopts the Chicago referencing style. However, the journal you’re submitting to may require a different reference style; specify the journal you're using with the class' \texttt{journal} option --- see lines 1--13 of \emph{sample.tex} for a list of options and instructions for selecting the journal. If you are using this template on Overleaf, Overleaf's build tool will automatically run \texttt{pdflatex} and \texttt{biber}. If you are compiling this template on your own local \LaTeX{} installation, please execute the following commands:
\begin{enumerate}
    \item \verb|pdflatex sample|
    \item \verb|biber sample|
    \item \verb|pdflatex sample|
    \item \verb|pdflatex sample|
\end{enumerate}

Lorem ipsum dolor sit amet, consectetur adipiscing elit, sed do eiusmod tempor incididunt ut labore et dolore magna aliqua. 

Lorem ipsum dolor sit amet, consectetur adipiscing elit, sed do eiusmod tempor incididunt ut labore et dolore magna aliqua. Lorem ipsum dolor sit amet, consectetur adipiscing elit, sed do eiusmod tempor incididunt ut labore et dolore magna aliqua. 


\subsection{Insert B head here}
Subsection text here. Lorem ipsum\autocite{Bayer_etal_2013} dolor sit amet, consectetur adipiscing elit, sed do eiusmod tempor incididunt ut labore\autocite{Adade_etal_2007} et dolore magna aliqua. 

 Lorem ipsum dolor sit amet, consectetur adipiscing elit, sed do eiusmod tempor incididunt ut labore et dolore magna aliqua. Lorem ipsum dolor sit amet, consectetur adipiscing elit, sed do eiusmod tempor incididunt ut labore et dolore magna aliqua. Lorem ipsum dolor sit amet, consectetur adipiscing elit, sed do eiusmod tempor incididunt ut labore et dolore magna aliqua. 

\subsubsection{Insert C head here}
Subsubsection text here. Lorem ipsum dolor sit amet, consectetur adipiscing elit, sed do eiusmod tempor incididunt ut labore et dolore magna aliqua. 
Lorem ipsum dolor sit amet, consectetur adipiscing elit, sed do eiusmod tempor incididunt ut labore et dolore magna aliqua. 

Lorem ipsum dolor sit amet, consectetur adipiscing elit, sed do eiusmod tempor incididunt ut labore et dolore magna aliqua. Lorem ipsum dolor sit amet, consectetur adipiscing elit, sed do\endnote{A footnote/endnote} eiusmod tempor incididunt ut labore et dolore magna aliqua. 

\section{Equations}

Sample equations. Lorem ipsum dolor sit amet, consectetur adipiscing elit, sed do eiusmod tempor incididunt ut labore et dolore magna aliqua. Lorem ipsum dolor sit amet, consectetur\endnote{Another footnote/endnote} adipiscing elit, sed do eiusmod tempor incididunt ut labore et dolore magna aliqua. Lorem ipsum dolor sit amet, consectetur adipiscing elit, sed do eiusmod tempor incididunt ut labore et dolore magna aliqua. 


%%% Numbered equation
\begin{equation}
\begin{aligned}\label{eq:first}
\frac{\partial u(t,x)}{\partial t} = Au(t,x) \left(1-\frac{u(t,x)}{K}\right)
 -B\frac{u(t-\tau,x) w(t,x)}{1+Eu(t-\tau,x)},\\
\frac{\partial w(t,x)}{\partial t} =\delta \frac{\partial^2w(t,x)}{\partial x^2}-Cw(t,x)
+D\frac{u(t-\tau,x)w(t,x)}{1+Eu(t-\tau,x)},
\end{aligned}
\end{equation}

 Lorem ipsum dolor sit amet, consectetur adipiscing elit, sed do eiusmod tempor incididunt ut labore et dolore magna aliqua. Lorem ipsum dolor sit amet, consectetur adipiscing elit, sed do eiusmod tempor incididunt ut labore et dolore magna aliqua. Lorem ipsum dolor sit amet, consectetur adipiscing elit, sed do eiusmod tempor incididunt ut labore et dolore magna aliqua. 

\begin{align}\label{eq:another}
\begin{split}
\frac{dU}{dt} &=\alpha U(t)(\gamma -U(t))-\frac{U(t-\tau)W(t)}{1+U(t-\tau)},\\
\frac{dW}{dt} &=-W(t)+\beta\frac{U(t-\tau)W(t)}{1+U(t-\tau)}.
\end{split}
\end{align}


%%%% Unnumbered equation
\begin{align*}
&\frac{\partial(F_1,F_2)}{\partial(c,\omega)}_{(c_0,\omega_0)} = \left|
\begin{array}{ll}
\frac{\partial F_1}{\partial c} &\frac{\partial F_1}{\partial \omega} \\\noalign{\vskip3pt}
\frac{\partial F_2}{\partial c}&\frac{\partial F_2}{\partial \omega}
\end{array}\right|_{(c_0,\omega_0)}\\
&\quad=-4c_0q\omega_0 -4c_0\omega_0p^2 =-4c_0\omega_0(q+p^2)>0.
\end{align*}


\section{Figures \& Tables}

The output for a single-column figure is in Figure~\ref{fig_sim}.  Lorem ipsum dolor sit amet, consectetur adipiscing elit, sed do eiusmod tempor incididunt ut labore et dolore magna aliqua. Lorem ipsum dolor sit amet, consectetur adipiscing elit, sed do eiusmod tempor incididunt ut labore et dolore magna aliqua. Lorem ipsum dolor sit amet, consectetur adipiscing elit, sed do eiusmod tempor incididunt ut labore et dolore magna aliqua. 

Lorem ipsum dolor sit amet, consectetur adipiscing elit, sed do eiusmod tempor incididunt ut labore et dolore magna aliqua. Lorem ipsum dolor sit amet, consectetur adipiscing elit, sed do eiusmod tempor incididunt ut labore et dolore magna aliqua. Lorem ipsum dolor sit amet, consectetur adipiscing elit, sed do eiusmod tempor incididunt ut labore et dolore magna aliqua. 

%See Figure~\ref{fig_wide} for a double-column figure; this is always at the top of a following page.


\begin{figure}[hbt!]
\centering
\includegraphics[width=0.75\linewidth]{example-image-16x10.pdf}
\caption{Insert figure caption here}
\label{fig_sim}
\end{figure}


\begin{figure*}
\centering
\includegraphics[width=0.8\linewidth]{example-image-16x10.pdf}
\caption{Insert figure caption here}
\label{fig_wide}
\end{figure*}


See example table in Table~\ref{table_example}.

\begin{table}[hbt!]
\begin{threeparttable}
\caption{An Example of a Table}
\label{table_example}
\begin{tabular}{llll}
\toprule
\headrow Column head 1 & Column head 2  & Column head 3 & Column head 4\\
\midrule
One\tnote{a} & Two&three three &four\\ 
\midrule
Three & Four&three three\tnote{b} &four\\
\bottomrule
\end{tabular}
\begin{tablenotes}[hang]
\item[]Table note
\item[a]First note
\item[b]Another table note
\end{tablenotes}
\end{threeparttable}
\end{table}


\section{Conclusion}
The conclusion text goes here.


\paragraph{Acknowledgments}
We are grateful for the technical assistance of A. Author.


\paragraph{Funding Statement}
This research was supported by grants from the <funder-name><doi>(<award ID>); <funder-name><doi>(<award ID>).

\paragraph{Competing Interests}
A statement about any financial, professional, contractual or personal relationships or situations that could be perceived to impact the presentation of the work --- or `None' if none exist

\paragraph{Data Availability Statement}
A statement about how to access data, code and other materials allowing users to understand, verify and replicate findings --- e.g. Replication data and code can be found in Harvard Dataverse: \verb+\url{https://doi.org/link}+.

\paragraph{Ethical Standards}
The research meets all ethical guidelines, including adherence to the legal requirements of the study country.

\paragraph{Author Contributions}
Please provide an author contributions statement using the CRediT taxonomy roles as a guide {\verb+\url{https://www.casrai.org/credit.html}+}. Conceptualization: A.A; A.B. Methodology: A.A; A.B. Data curation: A.C. Data visualisation: A.C. Writing original draft: A.A; A.B. All authors approved the final submitted draft.


%\endnote in some journals will behave like \footnote; and \printendnotes will not output anything. 
\printendnotes

\defbibnote{preamble}{By default, this template uses \texttt{biblatex} and adopts the Chicago referencing style. However, the journal you’re submitting to may require a different reference style; specify the journal you're using with the class' \texttt{journal} option --- see lines 1--13 of \emph{sample.tex} for a list of options and instructions for selecting the journal.}

\printbibliography[prenote={preamble}]
\appendix

\section{Example Appendix Section}

Lorem ipsum dolor sit amet, consectetur adipiscing elit, sed do eiusmod tempor incididunt ut labore et dolore magna aliqua. Lorem ipsum dolor sit amet, consectetur adipiscing elit, sed do eiusmod tempor incididunt ut labore et dolore magna aliqua. Lorem ipsum dolor sit amet, consectetur adipiscing elit, sed do eiusmod tempor incididunt ut labore et dolore magna aliqua. 

\end{document}