%%%%%%%%%%%%%%%%%%%%%%%%%%%%%%%%%%%%%%%%%%%%%%
% To select a journal, use its code for the 
% journal= option in the \documentclass command.
% The journal codes for this template are:
% 
% Annals of Actuarial Science: aas
% British Journal of Political Science: jps
% Network Science: nws
% Political Analysis: pla
% Political Science Research and Methods: ram
% Evolutionary Human Sciences: ehs
% Natural Language Processing: nlp
%%%%%%%%%%%%%%%%%%%%%%%%%%%%%%%%%%%%%%%%%%%%%%
\documentclass[
  journal=medium,
  manuscript=article-type,
  year=2026,
  volume=33
]{cup-journal}

\usepackage{amsmath}
\usepackage[nopatch]{microtype}
\usepackage{booktabs}

\title{Visión Artificial - Actividad 2: Exploración de filtros espaciales y morfológicos en escenarios reales}

\author{Orejón Blázquez, Álvaro}
\affiliation{Estudiantes, Universidad Internacional de La Rioja}
\email[Orejón Blázquez, Álvaro]{email@email.dom}

\author{Lopez Chamarro, Pau}
\affiliation{Estudiantes, Universidad Internacional de La Rioja}
\email[Lopez Chamarro, Pau]{email@email.dom}

\author{Iglesias Traviesa, Nuria}
\affiliation{Estudiantes, Universidad Internacional de La Rioja}
\email[Iglesias Traviesa, Nuria]{email@email.dom}

\author{Prieto Bailo, León Enrique}
\affiliation{Estudiantes, Universidad Internacional de La Rioja}
\email[Prieto Bailo, León Enrique]{contact@leonprieto.com}
% \author{T. Author}
% \affiliation{Second Division, Organization, City, Pincode, State, Country}

% \author{F.T. Author}
% \affiliation{Fourth Division, Organization, City, Pincode, State, Country}

\addbibresource{bibliography.bib}

\keywords{keyword entry 1, keyword entry 2, keyword entry 3} %% First letter not capped

\begin{document}

\begin{abstract}
Insert abstract text here. Lorem ipsum dolor sit amet, consectetur adipiscing elit, sed do eiusmod tempor incididunt ut labore et dolore magna aliqua. Lorem ipsum dolor sit amet, consectetur adipiscing elit, sed do eiusmod tempor incididunt ut labore et dolore magna aliqua. Lorem ipsum dolor sit amet, consectetur adipiscing elit, sed do eiusmod tempor incididunt ut labore et dolore magna aliqua. 
\end{abstract}

\section{Introducción}
\label{sec:intro}

El procesamiento digital de imágenes es una disciplina fundamental que permite mejorar, analizar e interpretar entornos reales. En campos donde se utilizan imágenes digitales, como la industria y la medicina, la aplicación de filtros
espaciales y morfológicos se ha convertido en una herramienta esencial para optimizar la calidad de las imágenes y extraer información relevante. Esto se debe a que las imágenes capturadas suelen estar afectadas por ruido, baja resolución y condiciones de iluminación,
lo que dificulta su análisis. La aplicación de estos filtros mejora la calidad visual de las imágenes, facilitando así su análisis y la extracción de información para la toma de decisiones.

Los filtros espaciales actúan modificando la intensidad de los píxeles en función de su vecindad, permitiendo así reducir ruido, suavizar regiones homogéneas y realzar o detectar bordes. Por otra banda, los filtros morfológicos actúan sobre la forma y estructura de los objetos mediante 
operaciones como la erosión, dilatación, apertura o cierre. Estas técnicas permiten eliminar elementos no deseados, separar o unir regiones y mejorar contornos.

Este trabajo presenta un enfoque multitemático que aplica, mediante un pipeline, un conjunto de filtros espaciales y morfológicos sobre imágenes de contextos médico, satelital e industrial. Este proceso permite comparar de forma directa los efectos de cada técnica dependiendo 
de la imagen utilizada. Además, se analiza cómo estas herramientas influyen en la calidad visual, la preservación de estructuras y la interpretabilidad de la información, identificando así las ventajas y limitaciones de cada filtro. De esta manera, se puede afirmar que 
no existe un filtro universalmente óptimo, y que la eficacia de cada técnica depende del contexto y de las características específicas de cada imagen.

\section{Material y métodos}
\label{sec:mat}

\subsection{Software utilizado}
\label{subsec:software}

El procesamiento de imágenes se realizó usando el lenguaje de programación \textbf{Python}. 
Por un lado, se empleó principalmente la librería \textbf{OpenCV} (\texttt{cv2}) para lectura, conversión de color y aplicación de los filtros espaciales y morfológicos, mientras que la librería \textbf{NumPy} se utilizó para operaciones de preprocesado y postprocesado. 
Por último, la visualización y comparación de resultados se llevó a cabo con \textbf{Matplotlib}.

\subsection{Selección de imágenes}
\label{sec:imagenes}

Con el objetivo de evaluar el comportamiento de filtros espaciales y operaciones morfológicas en un escenario multitemático, 
se trabajó con un conjunto acotado de imágenes reales seleccionadas de tres ámbitos diferenciados: \textit{industrial}, \textit{médico} y \textit{satelital/ambiental}. 

A la hora de elegir estas imágenes se han seleccionado ejemplos representativos con propiedades visuales claramente distintas, de manera que fuera posible comparar el efecto de las técnicas aplicadas en condiciones variadas.

En particular, se consideraron tres fuentes principales, asociadas a cada tipo de escenario:
\begin{itemize}
    \item \textbf{Open-i (escenario médico).} Open-i es un dataset de imágenes biomédicas mantenido por la \textit{National Library of Medicine}, que nos permitió obtener imágenes relacionadas con la temática biomédica. 
    En esta práctica se empleó para seleccionar radiografías, donde predominan gradientes suaves y estructuras anatómicas con bordes definidos, lo que resulta adecuado para analizar la atenuación de ruido mediante suavizado y los detectores de borde.

    \item \textbf{Science Source (escenario industrial).} Science Source es una biblioteca científica con imágenes relacionadas con tecnología, laboratorio e industria. 
    Se utilizó como fuente para seleccionar escenas industriales con patrones geométricos y contornos de alto contraste, idóneas para aplicar el filtrado espacial que afecta al detalle fino y las operaciones morfológicas, que pueden regular este tipo de estructuras.

    \item \textbf{EOSDA LandViewer / Sentinel Playground / Kaggle (escenario satelital y ambiental).} Se recurrió a plataformas de exploración y descarga de imágenes satelitales, que permiten obtener escenas de misiones a distintas resoluciones. 
    Estas fuentes proporcionan imágenes con grandes regiones homogéneas, texturas extensas (vegetación/cultivos) y elementos lineales (carreteras), lo que facilita contrastar el comportamiento de los filtros frente a variaciones de escala y contenido.
\end{itemize}

\subsection{Filtros espaciales}
\label{sec:esp}

\subsubsection{Suavizado}
\label{sec:suavizado}
\subsubsection{Realce de bordes}
\label{sec:rb}

\subsection{Filtros morfológicos}
\label{sec:morf}

\subsubsection{Dilatacion}
\label{sec:Dilatacion}
\subsubsection{Erosión}
\label{sec:erosion}
\subsubsection{Apertura}
\label{sec:ap}
\subsubsection{Cierre}
\label{sec:cierre}


\section{Resultados y métricas}
\label{sec:result}

\subsection{Escenario Industrial}
\subsection{Escenario Médico}
\subsection{Escenario Satelital}

\section{Conclusiones}
\label{sec:conc}

\section{Declaración de la IA}


























% ********* DEMO *********
\pagebreak
\pagebreak
\section{Insert A head here}
This demo file is intended to serve as a ``starter file''. It is for preparing manuscript submission only, not for preparing camera-ready versions of manuscripts. Manuscripts will be typeset for publication by the journal, after they have been accepted.

By default, this template uses \texttt{biblatex} and adopts the Chicago referencing style. However, the journal you’re submitting to may require a different reference style; specify the journal you're using with the class' \texttt{journal} option --- see lines 1--13 of \emph{sample.tex} for a list of options and instructions for selecting the journal. If you are using this template on Overleaf, Overleaf's build tool will automatically run \texttt{pdflatex} and \texttt{biber}. If you are compiling this template on your own local \LaTeX{} installation, please execute the following commands:
\begin{enumerate}
    \item \verb|pdflatex sample|
    \item \verb|biber sample|
    \item \verb|pdflatex sample|
    \item \verb|pdflatex sample|
\end{enumerate}

Lorem ipsum dolor sit amet, consectetur adipiscing elit, sed do eiusmod tempor incididunt ut labore et dolore magna aliqua. 

Lorem ipsum dolor sit amet, consectetur adipiscing elit, sed do eiusmod tempor incididunt ut labore et dolore magna aliqua. Lorem ipsum dolor sit amet, consectetur adipiscing elit, sed do eiusmod tempor incididunt ut labore et dolore magna aliqua. 


\subsection{Insert B head here}
Subsection text here. Lorem ipsum\autocite{Bayer_etal_2013} dolor sit amet, consectetur adipiscing elit, sed do eiusmod tempor incididunt ut labore\autocite{Adade_etal_2007} et dolore magna aliqua. 

 Lorem ipsum dolor sit amet, consectetur adipiscing elit, sed do eiusmod tempor incididunt ut labore et dolore magna aliqua. Lorem ipsum dolor sit amet, consectetur adipiscing elit, sed do eiusmod tempor incididunt ut labore et dolore magna aliqua. Lorem ipsum dolor sit amet, consectetur adipiscing elit, sed do eiusmod tempor incididunt ut labore et dolore magna aliqua. 

\subsubsection{Insert C head here}
Subsubsection text here. Lorem ipsum dolor sit amet, consectetur adipiscing elit, sed do eiusmod tempor incididunt ut labore et dolore magna aliqua. 
Lorem ipsum dolor sit amet, consectetur adipiscing elit, sed do eiusmod tempor incididunt ut labore et dolore magna aliqua. 

Lorem ipsum dolor sit amet, consectetur adipiscing elit, sed do eiusmod tempor incididunt ut labore et dolore magna aliqua. Lorem ipsum dolor sit amet, consectetur adipiscing elit, sed do\endnote{A footnote/endnote} eiusmod tempor incididunt ut labore et dolore magna aliqua. 

\section{Equations}

Sample equations. Lorem ipsum dolor sit amet, consectetur adipiscing elit, sed do eiusmod tempor incididunt ut labore et dolore magna aliqua. Lorem ipsum dolor sit amet, consectetur\endnote{Another footnote/endnote} adipiscing elit, sed do eiusmod tempor incididunt ut labore et dolore magna aliqua. Lorem ipsum dolor sit amet, consectetur adipiscing elit, sed do eiusmod tempor incididunt ut labore et dolore magna aliqua. 


%%% Numbered equation
\begin{equation}
\begin{aligned}\label{eq:first}
\frac{\partial u(t,x)}{\partial t} = Au(t,x) \left(1-\frac{u(t,x)}{K}\right)
 -B\frac{u(t-\tau,x) w(t,x)}{1+Eu(t-\tau,x)},\\
\frac{\partial w(t,x)}{\partial t} =\delta \frac{\partial^2w(t,x)}{\partial x^2}-Cw(t,x)
+D\frac{u(t-\tau,x)w(t,x)}{1+Eu(t-\tau,x)},
\end{aligned}
\end{equation}

 Lorem ipsum dolor sit amet, consectetur adipiscing elit, sed do eiusmod tempor incididunt ut labore et dolore magna aliqua. Lorem ipsum dolor sit amet, consectetur adipiscing elit, sed do eiusmod tempor incididunt ut labore et dolore magna aliqua. Lorem ipsum dolor sit amet, consectetur adipiscing elit, sed do eiusmod tempor incididunt ut labore et dolore magna aliqua. 

\begin{align}\label{eq:another}
\begin{split}
\frac{dU}{dt} &=\alpha U(t)(\gamma -U(t))-\frac{U(t-\tau)W(t)}{1+U(t-\tau)},\\
\frac{dW}{dt} &=-W(t)+\beta\frac{U(t-\tau)W(t)}{1+U(t-\tau)}.
\end{split}
\end{align}


%%%% Unnumbered equation
\begin{align*}
&\frac{\partial(F_1,F_2)}{\partial(c,\omega)}_{(c_0,\omega_0)} = \left|
\begin{array}{ll}
\frac{\partial F_1}{\partial c} &\frac{\partial F_1}{\partial \omega} \\\noalign{\vskip3pt}
\frac{\partial F_2}{\partial c}&\frac{\partial F_2}{\partial \omega}
\end{array}\right|_{(c_0,\omega_0)}\\
&\quad=-4c_0q\omega_0 -4c_0\omega_0p^2 =-4c_0\omega_0(q+p^2)>0.
\end{align*}


\section{Figures \& Tables}

The output for a single-column figure is in Figure~\ref{fig_sim}.  Lorem ipsum dolor sit amet, consectetur adipiscing elit, sed do eiusmod tempor incididunt ut labore et dolore magna aliqua. Lorem ipsum dolor sit amet, consectetur adipiscing elit, sed do eiusmod tempor incididunt ut labore et dolore magna aliqua. Lorem ipsum dolor sit amet, consectetur adipiscing elit, sed do eiusmod tempor incididunt ut labore et dolore magna aliqua. 

Lorem ipsum dolor sit amet, consectetur adipiscing elit, sed do eiusmod tempor incididunt ut labore et dolore magna aliqua. Lorem ipsum dolor sit amet, consectetur adipiscing elit, sed do eiusmod tempor incididunt ut labore et dolore magna aliqua. Lorem ipsum dolor sit amet, consectetur adipiscing elit, sed do eiusmod tempor incididunt ut labore et dolore magna aliqua. 

%See Figure~\ref{fig_wide} for a double-column figure; this is always at the top of a following page.


\begin{figure}[hbt!]
\centering
\includegraphics[width=0.75\linewidth]{example-image-16x10.pdf}
\caption{Insert figure caption here}
\label{fig_sim}
\end{figure}


\begin{figure*}
\centering
\includegraphics[width=0.8\linewidth]{example-image-16x10.pdf}
\caption{Insert figure caption here}
\label{fig_wide}
\end{figure*}


See example table in Table~\ref{table_example}.

\begin{table}[hbt!]
\begin{threeparttable}
\caption{An Example of a Table}
\label{table_example}
\begin{tabular}{llll}
\toprule
\headrow Column head 1 & Column head 2  & Column head 3 & Column head 4\\
\midrule
One\tnote{a} & Two&three three &four\\ 
\midrule
Three & Four&three three\tnote{b} &four\\
\bottomrule
\end{tabular}
\begin{tablenotes}[hang]
\item[]Table note
\item[a]First note
\item[b]Another table note
\end{tablenotes}
\end{threeparttable}
\end{table}


\section{Conclusion}
The conclusion text goes here.


\paragraph{Acknowledgments}
We are grateful for the technical assistance of A. Author.


\paragraph{Funding Statement}
This research was supported by grants from the <funder-name><doi>(<award ID>); <funder-name><doi>(<award ID>).

\paragraph{Competing Interests}
A statement about any financial, professional, contractual or personal relationships or situations that could be perceived to impact the presentation of the work --- or `None' if none exist

\paragraph{Data Availability Statement}
A statement about how to access data, code and other materials allowing users to understand, verify and replicate findings --- e.g. Replication data and code can be found in Harvard Dataverse: \verb+\url{https://doi.org/link}+.

\paragraph{Ethical Standards}
The research meets all ethical guidelines, including adherence to the legal requirements of the study country.

\paragraph{Author Contributions}
Please provide an author contributions statement using the CRediT taxonomy roles as a guide {\verb+\url{https://www.casrai.org/credit.html}+}. Conceptualization: A.A; A.B. Methodology: A.A; A.B. Data curation: A.C. Data visualisation: A.C. Writing original draft: A.A; A.B. All authors approved the final submitted draft.


%\endnote in some journals will behave like \footnote; and \printendnotes will not output anything. 
\printendnotes

\defbibnote{preamble}{By default, this template uses \texttt{biblatex} and adopts the Chicago referencing style. However, the journal you’re submitting to may require a different reference style; specify the journal you're using with the class' \texttt{journal} option --- see lines 1--13 of \emph{sample.tex} for a list of options and instructions for selecting the journal.}

\printbibliography[prenote={preamble}]
\appendix

\section{Example Appendix Section}

Lorem ipsum dolor sit amet, consectetur adipiscing elit, sed do eiusmod tempor incididunt ut labore et dolore magna aliqua. Lorem ipsum dolor sit amet, consectetur adipiscing elit, sed do eiusmod tempor incididunt ut labore et dolore magna aliqua. Lorem ipsum dolor sit amet, consectetur adipiscing elit, sed do eiusmod tempor incididunt ut labore et dolore magna aliqua. 

\end{document}